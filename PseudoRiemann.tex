\documentclass[aspectratio=169]{beamer}
\usetheme{metropolis}

\usepackage[T1]{fontenc}
\usepackage[utf8]{inputenc}
\usepackage{lmodern}

\setbeamertemplate{section in toc}{\textbullet\ \inserttocsection}
\setbeamertemplate{subsection in toc}{\hspace*{1.5em}\parbox[t]{0.75\textwidth}{\raggedright\footnotesize\textendash\ \inserttocsubsection}}

\title{Neural Geodesic Flows}
\subtitle{Extending to Pseudo-Riemannian Geometry}
\author{Mihir Talati (after J.~B\"urge, ETH Zurich 2025)}
\date{\today}

\begin{document}

\maketitle

\section{Overview}
\begin{frame}{Presentation Outline}
  \tableofcontents[subsectionstyle=show/show,subsubsectionstyle=hide]
\end{frame}

\begin{frame}
    \frametitle{Motivation}
    \centering
    \includegraphics[width=0.8\textwidth]{funny.png}
\end{frame}

%========================================================%
%  FOUNDATIONS & GEOMETRY
%========================================================%
\section{Foundations \& Geometry}

\begin{frame}{Classical NNs and the need for an ansatz}
  \begin{itemize}
    \item Universal approximation: NNs can fit many functions, but we need a \textbf{good ansatz} for stability and meaning.
    \item Our ansatz: observed dynamics are geodesic flows of a learned metric $\Rightarrow$ geometric invariants and structure.
  \end{itemize}
\end{frame}

\begin{frame}{Manifold hypothesis}
  \begin{itemize}
    \item Data in high-D $\mathbb{R}^n$ lie near a low-D manifold.
  \end{itemize}
  \centering\includegraphics[width=0.8\textwidth]{classifier.png}
\end{frame}

\begin{frame}{Geometry primer}
  \begin{itemize}
    \item Manifold + chart: local coordinates; tangent bundle $TM$ = positions + velocities.
    \item Geodesics: shortest paths for a metric; governed by the geodesic equation $\ddot x^k + \Gamma^{k}_{ab}\dot x^a\dot x^b=0$.
  \end{itemize}
\end{frame}

\begin{frame}{Tangent Planes}
  \centering\includegraphics[width=0.8\textwidth]{14.vector-bundle.png}
\end{frame}

\begin{frame}{Example}
  \centering\includegraphics[width=0.8\textwidth]{worldmap.png}
\end{frame}

\begin{frame}{Riemannian vs pseudo-Riemannian metrics}
  \begin{itemize}
    \item Riemannian: SPD metric, all spacelike (good for Euclidean-like systems).
    \item Pseudo-Riemannian: mixed signature (timelike + spacelike), needed for Lorentzian/relativistic systems.
    \item Extension: fixed-signature metric net + log-det regularizer; future: gating network to select signature automatically.
  \end{itemize}
\end{frame}

%========================================================%
%  MODEL & RESULTS
%========================================================%
\section{Model, Improvements, Results}

\begin{frame}{ResNets to NGF (condensed)}
  \begin{itemize}
    \item A (very) simplified ResNet block:
      \[
        h_{k+1} = h_k + f_\theta(h_k)
      \]
      where $k$ is the layer index.
    \item If you think of $k$ as discrete time and $f_\theta$ as a velocity field,
      this is exactly a forward Euler step for the ODE
      \[
        \frac{dh}{dt} = f_\theta(h(t)).
      \]
    \item Idea of \emph{Neural ODEs}: instead of stacking many discrete layers,
      treat depth as continuous time and let an ODE solver play the role of the network.
    \item Forward pass $\Rightarrow$ solve an ODE; backward pass $\Rightarrow$ differentiate
      \emph{through} the ODE solver (adjoint method, automatic differentiation, \dots). 
    \item NGF: a structured Neural ODE where the vector field is the geodesic flow of a learned metric $g$.
  \end{itemize}
\end{frame}

\begin{frame}{Neural Geodesic Flows}
  \centering\includegraphics[width=0.8\textwidth]{manifoldhyp.png}
\end{frame}

\begin{frame}{Encoder, metric, decoder + flow}
  \begin{itemize}
    \item Learn encoder $\psi_\theta$ (to $(x,v)$), decoder $\phi_\theta$ (back to data), metric $g_\theta(x)$.
    \item Baseline $g_\theta$: SPD via $I + L L^\top$.
    \item Extension $g_\theta$: fixed-signature pseudo-R (positive \emph{and} negative eigenvalues) + log-det regularization; future gating for signature.
  \end{itemize}
\end{frame}

\begin{frame}
    \frametitle{NGF forward pass (visual)}
    \centering
    \includegraphics[width=0.8\textwidth]{ngf.png}
\end{frame}

\begin{frame}{Changes (model \& loss)}
  \begin{itemize}
    \item \textbf{Metric network}: new pseudo-R module with fixed signature $(p,q)$; softplus scales with signs.
    \item \textbf{Conditioning}: \texttt{min\_diagonal}/\texttt{min\_scale} to keep $g$ non-degenerate near horizons.
    \item \textbf{Regularizer}: loss on $\log|\det g|$ to penalize near-singular metrics (configurable floor/weight).
    \item \textbf{Training}: PR runs use signatures $(1,1)$ or $(2,2)$, stronger metric regularization, gentler LR.
  \end{itemize}
\end{frame}

\begin{frame}{Why signature matters (physics alignment)}
  \begin{itemize}
    \item Lorentzian metrics encode causal structure (timelike vs spacelike).
    \item Learned eigenvalues at a probe point:
      \begin{itemize}
        \item Minkowski\_prngf: $[-0.12,\;0.29]$ (one timelike, one spacelike).
        \item AdS2\_prngf: $[-0.036,\;0.017]$ (Lorentzian, small curvature).
        \item RNGF baselines: both $>0$ (purely spatial).
      \end{itemize}
    \item Interpretation: mixed signs capture lightcones/horizons; SPD forces all spacelike $\Rightarrow$ distorted distances and rollout drift.
  \end{itemize}
\end{frame}

\begin{frame}{Findings \& next steps (pseudo-R vs R)}
  \begin{itemize}
    \item Findings: pseudo-R keeps correct signature on Lorentzian toy spacetimes (Minkowski, AdS2, Schwarzschild); Riemannian baselines drift on curved Lorentzian rollouts.
    \item Limitations: single chart; flat datasets don’t stress signature; Schwarzschild checkpoints need clean retrain.
    \item Next steps: gating network to select signature adaptively; atlas-based NGFs; stronger pseudo-R regularizers and long-horizon tests.
  \end{itemize}
\end{frame}

\end{document}
